\documentclass[10pt]{article}

%%%%    FORMATTING    %%%%
\usepackage{url}
\usepackage[utf8]{inputenc}
\usepackage{amsmath}
\usepackage{amssymb}
%\usepackage{wasysym}
\usepackage[version=4]{mhchem}
\usepackage{verbatim}
\usepackage{graphicx}
\usepackage[export]{adjustbox}
\usepackage[margin=1in]{geometry}
\usepackage[titletoc, title]{appendix}
\usepackage{hyperref}
\hypersetup{
    colorlinks = true,
    citecolor = red,
    linkcolor = blue,
    urlcolor = black,
}
%For code in appendix
\usepackage{listings}
\usepackage{color}
\definecolor{mygreen}{rgb}{0,0.6,0}
\definecolor{mygray}{rgb}{0.5,0.5,0.5}
\definecolor{mymauve}{rgb}{0.58,0,0.82}
\lstset{ %
  backgroundcolor=\color{white},   % choose the background color
  basicstyle=\footnotesize,        % size of fonts used for the code
  breaklines=true,                 % automatic line breaking only at whitespace
  captionpos=b,                    % sets the caption-position to bottom
  commentstyle=\color{mygreen},    % comment style
  escapeinside={\%*}{*)},          % if you want to add LaTeX within your code
  keywordstyle=\color{blue},       % keyword style
  stringstyle=\color{mymauve},     % string literal style
  frame = tb,
  numbers = left
}




\title{nEXO OD FLUKA Simulations Manual}
\author{Regan Ross}

%%%%    BEGIN DOCUMENT    %%%
\begin{document}

%%%%    Title Page    %%%
\begin{titlepage}
    \maketitle
    \vspace{4cm}
    \centering
    % \includegraphics*[scale=0.5, frame]{./figs/muons_det.png}
\end{titlepage}

\begin{abstract}
    This document is intented to provide insight into the functioning of the nEXO FLUKA simulations and justifications for choices made in their design. FLUKA can be a daunting software to work with given it is written in FORTRAN77, has sparse documentation, and whose source code, without a specific licence, is veiled behind a curtain. However, this proprietary simulation program is fast and rich with built-in features.
\end{abstract}


\vspace{1.5cm}
\listoffigures

\newpage
\tableofcontents

\break
%%%%%%%%%%%%%%%%%%%%%%%%%%%%%
%                           %
%%%%    INTRODUCTION     %%%%
%                           %
%%%%%%%%%%%%%%%%%%%%%%%%%%%%%
\part{Introduction}

% \clearpage
% % \appendixheaderon
% % \appendixpage
% % \begin{appendices}
    

% % \end{appendices}


% \newpage

% \section*{Acknowledgements}
% \paragraph{}


% \bibliographystyle{aip}
% \bibliography{manual.bib}

\end{document}