\section{Some Results}
\paragraph{}
The primary motivation for these FLUKA simulations was to validate the GEANT4 simulations that had previously been performed to estimate activation due to the cosmogenic muons. Other connections were investigated including the distances of closest approach (impact parameters) of the muons producing neutrons in the vicinity of the nEXO OD. There had been a connection established between the impact parameters and the propensity of a muon to produce neutrons within or entering the OD or the TPC volume; muons with large impact parameters were unlikely to contribute to activation.

\paragraph{}
For the simulations in this work, the impact parameters were calculated upon the random generation of the muons in the external python module. These data, along with all the other muon data, were saved and added to the ``.hdf5'' files after the simulations were performed. The activation data were scored with FLUKA's RESNUCLEi function that counts ``residual nuclei'' in a chosen region. The analyses were entirely performed externally using various python libraries.

\subsection{Activation}

\paragraph{}
The activation was scored for the TPC internal liquid xenon volume and its copper shell. Production rates were scored for a range of isotopes including xenon-137 and copper isotopes in the TPC shell. Over 150 simulated years, the mean yearly count of Xe-137 was 22 atoms. Figure \ref{fig:activation} displays tables of isotopes for both the TPC xenon region and the copper shell.

\begin{figure}[h]
    \begin{center}
    \includegraphics[scale=0.8]{figures/activation.png}
    \caption{A plot of the residual nuclei counted in their respective volumes. 150 years of data presented.}
    \label{fig:activation}
    \end{center}
\end{figure}

\subsection{Impact Parameters}

\paragraph{}
As mentioned previously, another parameter of interest with respect to the muons were the distances of closest approach; we refer to these as the impact parameters of the muons. The findings were particularly interesting. For the same 150-year batch of muons, all the muons responsible for neutrons in the TPC passed through the OD volume and would therefore be detectable. These results are shown in figure \ref{fig:impacts}. Bear in mind the maximum possible impact parameter for a muon in the OD is just shy of 9 meters, and the smallest impact parameter a muon can have without passing through the OD is the OD radius- 6.172 m. The maximum possible impact parameter from the simulations was nearly 12 meters; there are no muons beyond 11 meter impact parameters producing neutrons in the OD- let alone the TPC.

\begin{figure}[h]
    \begin{center}
    \includegraphics[scale=0.70]{figures/impacts.png}
    \caption{A plot of the the impact parameters of muons producing neutrons in either the TPC (left) or the OD (right)}
    \label{fig:impacts}
    \end{center}
\end{figure}
