\section{The nEXO OD Input File}
\paragraph{}
The input file (``.inp'') is exactly as it is named— it is the De facto interface between the user and the FLUKA binaries. It contains, among other things, the entire physical configuration of the media through which the particles are transported, the defining characteristics of the impinging beam (for simple sources), options for enabling particular physical processes, definitions of materials, and cards to deploy default FLUKA scoring methods. It is a human-readable ascii file with a very specific format. Each input card in the input file must not contain more than 8 fields each of which has a character limit. This is due (presumably) to constraints in the early development of FLUKA. Programmed in FORTRAN77 which imposes constrains on the length of statements which, in the early days, were written into computers (with very little memory) using paper punch cards. Now, this formatting is an annoying anachronism but it probably does still keep the program slim and fast. Generally though, a user need not worry about writing the input file directly, as there is a great GUI interface to FLUKA called \href{https://fluka.cern/documentation/examples/flair}{\textit{Flair}} which, by the way, is open source and contains all the possible input options \cite{Flair}. The next section will overview the specific components of the nEXO OD input file and the functions they serve. 

\paragraph{}
It is expected that the user interfaces with the input files via \href{https://fluka.cern/documentation/examples/flair}{\textit{Flair}} as this is probably the easiest way. Errors in the configuration will be unambiguously shown, and it is hard to mess up the format of the input file this way.